\documentclass[12pt]{article}
\usepackage[margin=1in]{geometry}
\usepackage{amsmath}
\usepackage[osf]{mathpazo}
\usepackage{microtype}
\usepackage[onehalfspacing]{setspace}
\usepackage{url}
\usepackage{booktabs}
\usepackage{fancyvrb}
\usepackage{xspace}
\def\php{\textsc{php}\xspace}
\def\ip{\textsc{ip}\xspace}
\def\dns{\textsc{dns}\xspace}
\newcounter{rules}
\newcommand\labeleq[1]{\refstepcounter{rules}\label{#1}}
\begin{document}
\VerbatimFootnotes
\title{Discouraging Wiki Spam with Proof-of-Work}
\author{Justin Bailey (\texttt{justinb@cs.pdx.edu}) \\ Dr.~Wu-chang Feng, CS510, Winter 2012}
\maketitle

My project uses a proof-of-work scheme to discourage wiki spam. I
implemented and tested a prototype to be used with the open-source
DokuWiki project. DokuWiki can be extended through plugins; my plugin
requires wiki authors to solve a computational problem before they can
submit content. I used several heuristics to determine the difficulty
of the puzzle presented to the author. The source code for my work can
be downloaded at \url{https://github.com/m4dc4p/dokuwiki-kapow}.

\section*{Wiki Software}

``Wiki'' software enables collaborative content creation and editing
on the internet. Authors do not need to use \textsc{html} to create
content, and articles written will generally be published immediately.
Wikipedia, one of the top 10 most visited sites on the
internet,\footnote{According to
  \url{http://www.google.com/adplanner/static/top1000}, accessed
  \today.} hosts over 3 million
articles,\footnote{From \url{http://en.wikipedia.org/wiki/Special:Statistics}.}
  all authored by volunteers and published using wiki software.

The popularity of some wiki sites and the ease with which content can
be created gives rise to ``wiki spam.'' Wiki spam consists
of links to spam sites, with no real content. By placing links to spam sites on various wikis, the
spammer attempts to raise their spam site's ranking in search engine
results,\footnote{See \url{http://en.wikipedia.org/wiki/Pagerank} for
  more.}  ultimately resulting in more traffic (and revenue) for the spammer.

Wiki software supports a variety of anti-spam techniques. The software
might only allow content creation from certain \ip addresses,
limit content creation to registered users, or require that all
content be approved by a moderator before being
published.\footnote{Though not specifically for wiki software, the
  article \url{http://en.wikipedia.org/wiki/Spam_in_blogs} discusses
  many of these techniques.} Each technique makes wiki spam harder
to create, but also makes it harder for legitimate authors to create content.

\section*{Proof of Work}
Client puzzles, or ``proof-of-work,'' attempt to thwart spam (and
other malicious behavior) by changing the economics of spam
creation. Proof-of-work requires authors to solve a
computationally difficult problem before their content will be
published. 

The difficulty of the puzzle can be scaled according to heuristics
that determine the ``spamminess'' of the particular author. For
non-spammers, the server gives an easy problem that can be solved
nearly instantaneously. For spammers, the server gives a problem that
may not be solved for years.

Spammers raise their search engine ranking by posting a large number
of links on different sites. The more links they can post, the more
their ranking will rise. Making it hard to post links changes the
economics for spammers --- if they cannot post links quickly and
cheaply, they will not make money. Therefore, by using
``proof-of-work'' to raise the cost for spammers, we can reduce the
profit potential of wiki spam.

\section*{DokuWiki}
The DokuWiki\footnote{Hosted at \url{http://www.dokuwiki.org}.}
project provides a free, open-source wiki implementation written in
\php. DokuWiki offers a number of mechanisms to combat spam: the
administrator of a given DokuWiki installation can choose to limit
content creation only to registered users; content can be blocked if
it contains certain words (such as ``Viagra''); external links can be
modified so they do not contribute to search engine ranking. DokuWiki
itself can by customized via plugins, an number of which offer other
anti-spam tools.\footnote{See
  \url{http://www.dokuwiki.org/plugins?plugintag=spam#extension__table}.}

\section*{A Proof of Work Plugin}
For my project, I created a proof-of-work plugin for
DokuWiki.\footnote{Source code available at
  \url{https://github.com/m4dc4p/dokuwiki-kapow}. I targeted the
  ``Angua'' (2012-01-25) version of DokuWiki. } My plugin extends
DokuWiki's editing page so that authors must solve a client puzzle
before the wiki software will accept their content updates. My plugin uses
several heuristics to determine the difficulty of the puzzle presented
to the author. 

I based my work on the guest book prototype developed by Tien Le,
Wu-chang Feng, and Ed Kaiser.\footnote{See
  \url{http://kapow.cs.pdx.edu/kapow/guestbook}.} They provide a \php
and JavaScript implementation of a hard cryptography problem with
adjustable difficulty and a
a \dns blacklisting heuristic, both of which I re-used entirely.

\section*{Heuristics}
My prototype uses the four heuristics shown in
Figure~\ref{fig_heuristics} to determine the puzzle's difficulty. Each
heuristic evaluates to an integer score. The plugin calculates
difficulty by raising $128$ to the sum of all the heuristic
scores:\footnote{I did not investigate the exponential behavior of the
  puzzle deeply, but scores above $16,384$ (i.e., $128^2$) result in
  puzzles that are not solved in the time I have patience for.}
\begin{align}
  \mathit{Dc} &= \lfloor128^{\mathit{score}}\rfloor\label{eq_dc}
\end{align}
\noindent where $Dc$ represents the difficulty.

\begin{figure}[t]
  \begin{center}
    \begin{tabular}{llcp{2in}}
      \multicolumn{2}{c}{\sc Heuristic}  & \sc Range & \hfil\sc Motivation \\\midrule\\
      \labeleq{eq_bl}\eqref{eq_bl} & \dns Blacklist & $0$ -- $255$ & {\raggedright Calculates spam ``threat'' of the author's \ip using \url{http://httpbl.org}.\par}\\
      \labeleq{eq_auth}\eqref{eq_auth} & User Logged In & $0$ -- $1$ & {\raggedright We give more trust to authenticated users.\par}\\
      \labeleq{eq_bc}\eqref{eq_bc} & Empty Breadcrumbs & $0$ -- $2$ & {\raggedright Spammers unlikely to browse the site, therefore breadcrumb trail will be empty.\par}\\
      \labeleq{eq_subnet}\eqref{eq_subnet} & \ip Not in Subnet & $0$ -- $1$ & {\raggedright Spammers not likely to be from the same subnet.\par}\\\bottomrule
    \end{tabular}
  \end{center}
  \caption{Heuristics used to determine the client puzzle's difficulty.}
  \label{fig_heuristics}
\end{figure}

Le, Feng and Kaiser implemented Rule~\eqref{eq_bl}, a \dns
``blacklist'' look-up that takes a client \ip and determines its
threat level. $Dc$ quickly blows up with even the least threatening
{\ip}s. In practice, this heuristic may be too aggressive. If so, it
could be scaled to give a result between $0$ and $3$ rather than $0$
and $255$.

Rule~\eqref{eq_auth} checks if the author authenticated themselves. If
so, it evaluates to $0$, otherwise $1$. This rule assumes that a
spammer will not bother to register an account before posting spam.

The breadcrumb heuristic, Rule~\eqref{eq_bc}, checks if the author
navigated anywhere else on the wiki. If the author browsed at least
two other pages, the rule evaluates to $0$. Otherwise, the rule calculates $2$
minus the number of pages navigated. This rule assumes a spammer will
go to one page and begin posting spam, whereas a normal user would
browse the wiki before adding content.

Rule~\eqref{eq_subnet} checks if the author browses from the same
subnet as the wiki. This rule evaluates to $0$ when the author
originates from the same subnet; otherwise, it evaluates to $1$. This
rule gives more trust to local traffic. For a wiki on the open
internet it would probably not be appropriate.

Alone, none of these rules (except the \dns blacklist) will generate a
very difficult puzzle. However, if the author fails any three of the
rules, they will find they effectively cannot post content.

\section*{Discouraging Spam vs.\ Detecting Spam}
Early on I attempted to use the ``spam'' score calculated by
SpamAssassin\footnote{Found at \url{http://spamassassin.apache.org}.}
to determine if the content submitted was spam or not. This approach
came directly from the webmail prototype of Le, Feng and
Spencer.\footnote{See \url{http://kapow.cs.pdx.edu/kapow/mail}.}
Necessarily, authors needed to submit content they could be
assigned a puzzle. In contrast, I setled on an approach that
assigns a puzzle to each author immediately, as they compose content.\footnote{The discussion
found at \url{http://wiki.apache.org/spamassassin/BlogSpamAssassin} inspired
my approach.}

This approach attempts to discourage spam from being posted at
all. When an author begins to create or edit a page, my plugin uses
the heuristics given above and assigns a puzzle to the author. Until
the puzzle is solved, the author cannot submit their
content. Legitimate authors will probably take a little time composing
their content, so they may not even notice the wait. Even a legitimate
user given a hard puzzle may not mind so much if they only create or
edit one piece of content. Spammers, on the other hand, will be discouraged from
posting content and editing a large number of pages.

In contrast, the detection approach makes all users wait to solve a
puzzle. The system will only generate a puzzle after the author
submits their content. The author must then wait to solve the puzzle
before they can proceed. While this approach may do a better job of
detecting spam in the long run, I believe the legitimate authors pay
too high a price.

\section*{Future Work}

My prototype only implements the most rudimentary detection rules, and
does not lend itself to re-use by other plugin authors. DokuWiki
supports an event system by which plugins can communicate and
react. My plugin could expose events for calculating the difficulty
score, allowing other plugins to implement their own heuristics for
detecting spammers. 

My plugin always (and only) modifies DokuWiki's editing
form. Proof-of-work might be useful elsewhere in the site, though. The
plugin could be architect ed to expose its proof-of-work
implementation for re-use in other contexts.

\section*{Conclusion}
My project addresses the problem of wiki spam. I created a plugin for
the open-source DokuWiki project. My plugin used a proof-of-work
scheme to discourage spam content from being posted. The plugin could be
extended in a number of ways, such as by implementing an event-based
system for extending the heuristics used to determine puzzle
difficulty. Source code for the plugin can be downloaded from
\url{https://github.com/m4dc4p/dokuwiki-kapow}.

\end{document}
